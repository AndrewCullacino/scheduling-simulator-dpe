% compile with XeLaTex or LuaLaTeX (this is due to selection of specific fonts)
% on overleaf: Menu > Settings > Compiler

\documentclass[final,hyperref={pdfpagelabels=false}]{beamer}
\usepackage{tangocolors}
\mode<presentation>
{
  %\usetheme{Berlin}
  \usetheme{Algo}
}
\usepackage{amsmath,amsthm, amssymb, latexsym}
\boldmath
\usepackage[english]{babel}
%\usepackage[utf8]{inputenc}
\usepackage[orientation=portrait,size=a1,scale=1.4,debug]{beamerposter}

%%%%%%%%%%%%%%%%%%%%%%%%%%%%%%%%%%%%%%%%%%%%%%%%%%%%%%%%%%%%%%%%%%%%%%%%%%%%%%%%%5
\graphicspath{{../simulator/visualizations/}}

\title{Greedy Scheduling with Dynamic Priority Elevations and Deadlines}
\author{Jintian Wang (z5536837) \and Dennis Shu (z5522609) \and Evan Lin (z5589313)}
\institute{Mentor: Ayda Valinezhad Orang\\
Group: 5}
\date{}

%%%%%%%%%%%%%%%%%%%%%%%%%%%%%%%%%%%%%%%%%%%%%%%%%%%%%%%%%%%%%%%%%%%%%%%%%%%%%%%%%5
\begin{document}
\begin{frame}{}
  \vfill
  \begin{columns}[t]
    \begin{column}{.5\linewidth}
      \begin{mybox}{Summary}
        \large
        \textbf{Key Findings from 14 scenarios, 84 experimental runs:}
        \begin{itemize}
          \item DPE provides \textbf{no advantage} over baselines (0/14 scenarios)
          \item EDF: best overall performer (16.7\% better in deadline-critical scenarios)
          \item SPT: best for batch processing (7.1\% better makespan)
          \item \textbf{Resource provisioning dominates} algorithmic sophistication (79\% of scenarios)
        \end{itemize}
      \end{mybox}

      \begin{myindigobox}{Problem \& Motivation}
        \large
        \textbf{Problem:} Parallel machine scheduling with priority classes and deadlines
        \begin{itemize}
          \item \textbf{Tasks:} $n$ indivisible tasks with processing times $p_i$, arrival times $a_i$
          \item \textbf{Priorities:} High priority (deadline $D_h$) and low priority (deadline $D_l$)
          \item \textbf{Machines:} $m$ identical parallel machines (non-preemptive)
          \item \textbf{Objective:} Minimize makespan while meeting deadlines
        \end{itemize}

        \vspace{1em}
        \textbf{Challenge:} Traditional priority scheduling can starve low-priority tasks
      \end{myindigobox}

      \begin{myindigobox}{Algorithms Tested}
        \large
        \textbf{Three Baseline Algorithms:}
        \begin{itemize}
          \item \textbf{SPT:} Shortest Processing Time First
          \item \textbf{EDF:} Earliest Deadline First (deadline-aware)
          \item \textbf{Priority-First:} High priority tasks scheduled before low priority
        \end{itemize}

        \vspace{1em}
        \textbf{Novel Algorithm:}
        \begin{itemize}
          \item \textbf{DPE:} Dynamic Priority Elevation
          \item Elevates low-priority tasks when deadline pressure exceeds threshold $\alpha$
          \item Tested with $\alpha \in \{0.5, 0.7, 0.9\}$
          \item Deadline pressure: $\frac{t - a_i}{D_l - a_i}$
        \end{itemize}

        \vspace{1em}
        \textbf{Complexity:} All algorithms $O(n \log n)$ time, $O(n+m)$ space
      \end{myindigobox}

      \begin{myredbox}{Critical Discovery: EDF Bug Fixed}
        Initial implementation processed events incorrectly, causing EDF to behave like SPT.

        \vspace{0.3em}
        \textbf{Impact after fix:}
        \begin{itemize}
          \item Extreme 1: 85.7\% → \textbf{100.0\%} success
          \item Extreme 3: 75\% → \textbf{100\%} success, makespan 12 → \textbf{10} (16.7\%)
        \end{itemize}

        \vspace{0.3em}
        \alert{Lesson: Implementation quality matters more than algorithmic sophistication}
      \end{myredbox}

    \end{column}
    \begin{column}{.5\linewidth}
      \begin{mybox}{Comprehensive Results}
        \centering
        \includegraphics[width=0.95\textwidth]{makespan_comparison_detailed.png}

        \vspace{0.5em}
        \raggedright
        \textbf{Overall Performance (14 scenarios):}
        \begin{itemize}
          \item \textbf{EDF:} Best avg makespan (12.1), highest success (91.7\%)
          \item \textbf{SPT:} Won 1/14 (Batch Arrival, 7.1\% better)
          \item \textbf{DPE:} Never outperformed baselines (0/14)
          \item \textbf{11/14 scenarios:} All algorithms identical
        \end{itemize}
      \end{mybox}

      \begin{mygreenbox}{Example: Batch Arrival (SPT Advantage)}
        \centering
        \textbf{SPT achieves 7.1\% better makespan}

        \includegraphics[width=0.48\textwidth]{Batch_Arrival_SPT_gantt.png}
        \includegraphics[width=0.48\textwidth]{Batch_Arrival_EDF_gantt.png}

        \raggedright
        \vspace{0.3em}
        SPT (left): makespan=13 | EDF (right): makespan=14
      \end{mygreenbox}

      \begin{mygreenbox}{Example: Extreme 3 (EDF Advantage)}
        \centering
        \textbf{EDF achieves 16.7\% better makespan, 100\% vs 75\% success}

        \includegraphics[width=0.48\textwidth]{Extreme_3_SPT_gantt.png}
        \includegraphics[width=0.48\textwidth]{Extreme_3_EDF_gantt.png}

        \raggedright
        \vspace{0.3em}
        SPT (left): 75\% success, makespan=12 | EDF (right): 100\% success, makespan=10
      \end{mygreenbox}

      \begin{mybox}{Conclusions \& Practical Recommendations}
        \textbf{Key Takeaways:}
        \begin{itemize}
          \item \textbf{Negative result has value:} DPE does not improve performance
          \item \textbf{Context-specific selection:} Use SPT for batch, EDF for deadlines
          \item \textbf{Resource provisioning first:} Adequate capacity matters more than algorithms
        \end{itemize}

        \vspace{0.5em}
        \textbf{Practical Guidance:}
        \begin{itemize}
          \item Provision resources adequately before optimizing algorithms
          \item Choose proven baselines matched to workload (SPT/EDF)
          \item Prioritize implementation quality and rigorous testing
        \end{itemize}
      \end{mybox}

      \begin{mybox}{References}
        \scriptsize
        \begin{enumerate}
          \item Alves de Queiroz, T., Figueira, G., Pereira, M., et al. (2023). Dynamic patient scheduling with priority classes and time windows in an emergency department. \textit{Operations Research for Health Care}, 100.
          \item Graham, R.L. (1966). Bounds for certain multiprocessing anomalies. \textit{Bell System Technical Journal}, 45, 1563-1581.
          \item Kahraman, C., Engin, O., Kaya, İ., \& Öztayşi, B. (2010). Multiprocessor task scheduling in multistage hybrid flow-shops: A parallel greedy algorithm approach. \textit{Applied Soft Computing}, 10(4), 1293-1300.
          \item Lee, C.Y., \& Pinedo, M. (1997). Scheduling jobs on parallel machines with sequence-dependent setup times. \textit{European Journal of Operational Research}, 100(3), 464-474.
          \item Liu, C.L., \& Layland, J.W. (1973). Scheduling algorithms for multiprogramming in a hard-real-time environment. \textit{Journal of the ACM}, 20(1), 46-61.
          \item Zhang, Q., Zhani, M.F., Zhang, S., et al. (2020). Dynamic energy-aware capacity provisioning for cloud computing environments. \textit{Future Generation Computer Systems}, 31-45.
        \end{enumerate}
      \end{mybox}
    \end{column}
  \end{columns}
  \vfill
\end{frame}
\end{document}
