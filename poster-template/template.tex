% compile with XeLaTex or LuaLaTeX (this is due to selection of specific fonts)
% on overleaf: Menu > Settings > Compiler

\documentclass[final,hyperref={pdfpagelabels=false}]{beamer}
\usepackage{tangocolors}
\mode<presentation>
{
  %\usetheme{Berlin}
  \usetheme{Algo}
}
\usepackage{amsmath,amsthm, amssymb, latexsym}
\boldmath
\usepackage[english]{babel}
%\usepackage[utf8]{inputenc}
\usepackage[orientation=portrait,size=a1,scale=1.4,debug]{beamerposter}

%%%%%%%%%%%%%%%%%%%%%%%%%%%%%%%%%%%%%%%%%%%%%%%%%%%%%%%%%%%%%%%%%%%%%%%%%%%%%%%%%5
%\graphicspath{{figures/}}

\title{\LaTeX{} Template for COMP3821/9801 Posters with overflowing title}
\author{group member 1 (zID1) \and group member 2 (zID2) \and group member 3 (zID3) \\ group member 4 (zID4) \and group member 5 (zID5) \and group member 6 (zID6)}
\institute{Mentor: mentor name\\
Group: group number}
\date{}

%%%%%%%%%%%%%%%%%%%%%%%%%%%%%%%%%%%%%%%%%%%%%%%%%%%%%%%%%%%%%%%%%%%%%%%%%%%%%%%%%5
\begin{document}
\begin{frame}{}
  \vfill
  \begin{columns}[t]
    \begin{column}{.5\linewidth}
      \begin{mybox}{Summary}
        This is a \textbf{primary} color box.

        \centering
        {\tiny tiny}\par
        {\scriptsize scriptsize}\par
        {\footnotesize footnotesize}\par
        {\normalsize normalsize}\par
        {\large large}\par
        {\Large Large}\par
        {\LARGE LARGE}\par
        {\huge huge}\par
        {\Huge Huge}\par
        {\veryHuge VeryHuge}\par
        {\VeryHuge VeryHuge}\par
        %{\VERYHuge VERYHuge}\par
      \end{mybox}

      \begin{myindigobox}{A regular box}
        \begin{itemize}
          \item \texttt{a monospace item}
          \item some items
          \item some items
          \item some items
        \end{itemize}
      \end{myindigobox}

      \begin{myredbox}{Pay attention!}
        This is important.
        \begin{itemize}
          \item some inline maths $\alpha=\gamma, \sum_{i}$
          \item and displayed maths
        \end{itemize}
        $$\alpha=\gamma, \sum_{i}$$
      \end{myredbox}

      \begin{mygreenbox}{Example}
        This is an example.
      \end{mygreenbox}

      \begin{mypinkbox}{Theorem (My Amazing Theorem)}
        This is not a poster.
      \end{mypinkbox}
      % also available: mytealbox, mybluebox, mypurplebox, myorangebox
    \end{column}
    \begin{column}{.5\linewidth}
      \begin{mybox}{Introduction}
        \begin{itemize}
          \item some items and $\alpha=\gamma, \sum_{i}$
          \item some items
          \item some items
          \item some items
        \end{itemize}
        $$\alpha=\gamma, \sum_{i}$$
      \end{mybox}

      \begin{mypinkbox}{Definition (circular definition)}
        A \alert{circular definition} is a circular definition.
      \end{mypinkbox}

      \begin{mygreenbox}{Example}
        \begin{itemize}
          \item some items and $\alpha=\gamma, \sum_{i}$
          \item some items
          \item some items
          \item some items
        \end{itemize}
        $$\alpha=\gamma, \sum_{i}$$
      \end{mygreenbox}

      \begin{myindigobox}{Note}
        This is a note.
      \end{myindigobox}

      \begin{center}
        % Petersen graph with a 3-coloring
        \begin{tikzpicture}[scale=3]
          \tikzstyle{vertex}=[circle,fill=black,inner sep=0mm,minimum size=3ex]
          \node[vertex,fill=unsw@indigo]  (1) at (18:1cm) {};
          \node[vertex,fill=unsw@green]  (2) at (90:1cm) {};
          \node[vertex,fill=unsw@red]  (3) at (162:1cm) {};
          \node[vertex,fill=unsw@red]  (4) at (234:1cm) {};
          \node[vertex,fill=unsw@indigo]  (5) at (306:1cm) {};
          \node[vertex,fill=unsw@green]  (6) at (18:2cm) {};
          \node[vertex,fill=unsw@red]  (7) at (90:2cm) {};
          \node[vertex,fill=unsw@green]  (8) at (162:2cm) {};
          \node[vertex,fill=unsw@indigo]  (9) at (234:2cm) {};
          \node[vertex,fill=unsw@red] (10) at (306:2cm) {};

          \draw[line width=5pt,color=unsw@black] (1)--(3)--(5)--(2)--(4)--(1)
          (6)--(7)--(8)--(9)--(10)--(6)
          (1)--(6) (2)--(7) (3)--(8) (4)--(9) (5)--(10);
        \end{tikzpicture}
      \end{center}

      \begin{mybox}{Conclusion}
        This is the conclusion.

      \end{mybox}
      \begin{mybox}{References}

      \end{mybox}
    \end{column}
  \end{columns}
  \vfill
\end{frame}
\end{document}
