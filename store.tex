\documentclass[oneside,12pt,english]{article}
\usepackage{amsmath,amsfonts,stmaryrd,amssymb, enumerate, amsthm, mathtools}
\newtheorem{theorem}{Theorem}
\newtheorem{lemma}{Lemma}
\newtheorem{corollary}{Corollary}
\newtheorem{proposition}{Proposition}
\theoremstyle{definition}
\newtheorem{definition}{Definition}
\usepackage{algorithm}
\usepackage{babel}
\usepackage[utf8]{inputenc}
\usepackage{color}
\usepackage{hyperref}
\usepackage{geometry, parskip}
\geometry{
	paper=a4paper,
	top=2.5cm,
	left=2.5cm,
	right=2.5cm,
}
\usepackage[utf8]{inputenc}
\usepackage[T1]{fontenc}
\usepackage{XCharter, enumitem}
% solution box (simplified)
\usepackage[most]{tcolorbox}

% Additional packages for professional appearance
\usepackage{titling}
\usepackage{fancyhdr}
\usepackage{xcolor}
\usepackage{tikz}
\usepackage{graphicx}

% Define colors
\definecolor{darkgray}{RGB}{64,64,64}

% Configure hyperref colors
\hypersetup{
    colorlinks=true,
    linkcolor=black,
    citecolor=black,
    urlcolor=black,
    pdfborderstyle={/S/U/W 1}
}


% Custom author formatting
\preauthor{%
  \begin{center}
  \vspace{3em}
    {\Large\textbf{Project Team}}\\[2em]
    \begin{tabular}{c}
      {\large Jintian Wang (z5536837)} \\[0.5em]
      {\large Dennis Shu (z5522609)} \\[1em]
      \large\textbf{Mentor:} {\large Ayda Valinezhad Orang}
    \end{tabular}
  \end{center}
}
\postauthor{}

% Custom date formatting
\predate{%
  \begin{center}
  \vfill
  {\normalsize School of Computer Science and Engineering}\\
  {\normalsize University of New South Wales}\\
  {\normalsize Sydney, Australia}
  \end{center}
}
\postdate{}

% reference
\bibliography{references}
\bibliographystyle{plain} 

% Configure fancy headers for subsequent pages
\pagestyle{fancy}
\fancyhf{}
\fancyfoot[C]{\thepage}
\renewcommand{\headrulewidth}{0pt}
\renewcommand{\footrulewidth}{0pt}

% Customize section headers
\usepackage{titlesec}
\titleformat{\section}
  {\Large\bfseries}
  {\thesection}
  {1em}
  {}
  [\vspace{-0.5em}\rule{\textwidth}{1pt}]

\title{Greedy Scheduling with Deadlines and Priorities}
\author{}
\date{}

\begin{document}

% Create the title page
\begin{titlepage}
\maketitle
\thispagestyle{empty}
\end{titlepage}

% Reset page numbering and apply fancy headers
\setcounter{page}{1}
\thispagestyle{fancy}

% Table of contents with custom formatting
\renewcommand{\contentsname}{\Large Table of Contents}
\tableofcontents
\newpage

\section{Introduction}
\label{sec:introduction}
Scheduling tasks on parallel machines is a fundamental problem in computer science with applications in cloud computing, manufacturing, and network management. We study fair scheduling algorithms that must balance efficiency (minimizing total completion time) with limited waiting time guarantees for different priority classes.

% ===== WEEK 3: PROPOSAL [15%] =====
\section{Project Proposal}
\subsection{Problem Statement}
We consider the problem of scheduling tasks on parallel machines where tasks have different priority levels requirements.
\newline
\begin{definition}[Problem Instance]
    An instance consists of:
    \begin{itemize}
        \item A set of $n$ indivisible tasks $\mathcal{T} = \{T_1, T_2, \ldots, T_n\}$
        \item A set of $m$ identical parallel machines $\mathcal{M} = \{M_1, M_2, \ldots, M_m\}$
        \item Each task $T_i$ has:
        \begin{itemize}
            \item Processing time $p_i > 0$
            \item Priority class $c \in \{h, l\}$ where $h$ is high priority and $l$ is low priority
        \end{itemize}
        \item Machine properties:
        \begin{itemize}
            \item A machine can only process $1$ task at a time, but can process arbitrarily many tasks sequentially
            \item Each task is assigned to exactly one machine (indivisible - no task splitting)
            \item Once the machine begins to process a task $T_i$, it works without interruption until completing the task.
        \end{itemize}
        \item Deadline constraints
        \begin{itemize}
            \item All tasks $T_i$ have deadlines $D_h > 0$ or $D_l > 0$
            \item $D_h < D_l$, where high priority tasks have earlier deadlines
        \end{itemize}
    \end{itemize}
\end{definition}

\begin{definition}[Completion Time]
For a task $T_i$ with arrival time $a_i$ and start time $s_i \geq a_i$, the completion time is:
\[
C_i = s_i + p_i
\]
\end{definition}

\begin{definition}[Feasible Schedule]
    A schedule is \emph{feasible} if 
    \begin{itemize}
        \item For every high-priority task $T_i$:$C_i \leq D_h$
        \item For every low-priority task $T_j: C_j \leq D_l$
    \end{itemize}
    
\end{definition}
    
\begin{definition}[Makespan]
The \emph{makespan} of a schedule is the time when all tasks have finished, which is equivalent to the the completion time of the last task:
\[
C_{\max} = \max_{i \in [n]} C_i
\]
\end{definition}
    
\textbf{Objective:} Design an algorithm to find a feasible schedule that minimizes the makespan $C_{\max}$.
    
\subsection{Survey}
    \subsubsection{Known Results} Graham has established a theoretical framework for multiprocessor scheduling problems. He studied scheduling $n$ tasks on $m$ identical machines where tasks have processing times and precedence constraints, and the goal is to minimise makespan.
    His results provide baseline for our problem, which is that any scheduling algorithm achieves a makespan $\omega \leq (2 - \frac{1}{m}) \times\omega_0$, where $\omega_0$ is the optimal makespan and $m$ is the number of identical machines \cite{graham1966bounds}.
    \newline
    Therefore, we expect our algorithm should achieve better than $2 \times \omega_0$ makespan.
    
    \subsubsection{Motivations}
    
    \subsubsection{Special Cases}
    \textit{Special cases where you aim to achieve or have thought about.}
    
    \subsubsection{Preliminary Direction}
    
\subsection{Research Plan}
\textit{A set of goals you aim to achieve by Week 7 (i.e. progression check) and a set of goals you aim to achieve by Week 10. You should aim to keep your research plan realistic to the confines of the project timeline.}

% waiting to be written later
% \subsection{Ideas That Might Not Work}
%     \textbf{REQUIRED: 1-2 approaches that probably won't work and why:}
    
%     \textbf{Idea 1:} [Describe approach]
%     \begin{itemize}
%         \item Why it seems reasonable: [Initial appeal]
%         \item Why it won't work: [Fundamental problems]
%     \end{itemize}

% ===== WEEK 7: PROGRESS CHECK [15%] =====
% Uncomment for Week 7 submission
% \section{Progress Report}
% 
% \subsection{Problem Summary}
%     \textbf{Brief restatement of the problem from proposal}
% 
% \subsection{Progress Update}  
%     \textbf{What have you accomplished since the proposal?}
%     % For THEORETICAL: Current results + new results + proofs + plan progress
%     % For EXPERIMENTAL: Algorithm implementation + empirical results
%     % For SCHOLARLY: Historical outline + new discoveries
%     % For CREATIVE: Current project state

% ===== WEEK 10: FINAL REPORT [70%] =====
% Uncomment for final submission
% \section{Final Report}
% 
% \subsection{Introduction}
%     \textbf{Problem motivation and overview}
%
% \subsection{Background and Related Work}
%     \textbf{Extended literature review}
%
% \subsection{Methodology}
%     \textbf{Detailed description of your approach}
%
% \subsection{Results}
%     \textbf{Your findings - theoretical results, experimental data, implementations}
%
% \subsection{Discussion}
%     \textbf{Analysis and interpretation of results}
%
% \subsection{Conclusion and Future Work}
%     \textbf{Summary and next steps}

% ===== BIBLIOGRAPHY =====
\newpage
\renewcommand{\refname}{\Large References}
\bibliographystyle{plain}
\bibliography{references}

\end{document}